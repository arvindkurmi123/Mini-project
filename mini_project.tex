\documentclass[]{article}

\usepackage[hmargin=3.5cm, vmargin=3.5cm]{geometry}
\usepackage{fancyhdr}
\usepackage[utf8]{inputenc}
\usepackage{listings}
\usepackage{graphicx}

\title{}
\author{}
\date{}
\fancypagestyle{first}
{
    \fancyhead[L]{{0801CS211023}}    
    \fancyhead[C]{\fontsize{30}{10}\selectfont\textbf{LATEX REPORT}}
     \fancyhead[R]{{Arvind Kurmi }}
}
\fancypagestyle{others}
{
    \fancyhead[L]{\small {Roll No. 0801CS211023}}    
   
     \fancyhead[R]{\small {NAME : Arvind kurmi}}
}
\thispagestyle{first}
\begin{document}

\vspace*{1mm}

\section{Objective of Project-}

\paragraph{}\large To create a program to create a simple trading account and perform some functions of trading with it where customer will be given Rupees 10000.

\vspace*{1mm}

\section{Function Description-}

\paragraph{Function(1):}\textbf{Account\_Creater}:  This function creates a new account of customer if account do not exist already and return the account number in both cases either account exists or not.		
														
\paragraph{Function(2):}\textbf{Main\_Options}:  This function shows all the options available to use for the user or customer. 
											
\paragraph{Function(3):}\textbf{Vyapaar}:  This function switches the user to different menus of the different type of Options and uses some inputs and external methods in it.			
				
\paragraph{Function(4): }\textbf{Print\_Tradings}:  This function shows different options for Trading to the user.

\paragraph{Function(5): }\textbf{buy}:  This buy function is used for buying new stocks and it has the exception to not to have the sufficient money to buy the stocks and decrement the money from the account of the user . This function uses different local methods that are set\_balance, get\_balance, set\_stocks, get\_stocks etc. 

\paragraph{Function(6): }\textbf{sell}:  This sell function is used to sell the stocks the customer already has. and increment the money in his account respectively. it also uses some local methods like showing how many stocks the customer has and how to decrement the stocks form the account.

\paragraph{Function(7) : }\textbf{Print\_DataSheet}: This function is used to print the information of the customer. 

\paragraph{Function (8) : }\textbf{Deposit}:  This function is used to deposit the money into the account of the user having the exception of deposit limit exception. 

\paragraph{Function (9) : }\textbf{Withdraw}:  This function is used to withdraw the money from the account of the user and decrement it from the account of the user.
\pagebreak

\thispagestyle{others}
\paragraph{Function (10) : }\textbf{Check\_Pin}:  This function is used to check the pin if it is same as the set pin or not. 

\paragraph{Other Functions : } There are many local methods to set, get or print the values of the local and global variables .

\vspace*{1mm}
\section{code of program}

\scriptsize\begin{lstlisting}[language=java, caption= Java Mini Project ]

// importing packages
import java.util.*;

// Main class
public class Mini_project {

    // main function 
    public static void main(String[] args) {
        // creating object of Scanner class
        Scanner sc = new Scanner(System.in);

        /*creating an array of objects of Trading class
         * and passing it to account creater
         * and after the registration passing it to the Vyapaar method
         * for the trading purpose
        */
        Trading t[] = new Trading[20];
        int account_no_counter = 0;

        System.out.println("*****TRADING ACCOUNT*****");
        System.out.println("Hello sir, \nWelcome to the trading account");

        account_no_counter = acCreater(account_no_counter, t);

        // starting trading 
        MainOptions();
        Vyapaar(sc.nextInt(), t[account_no_counter]);


    }

    // for printing MainOptions
    public static void MainOptions() {
        System.out.println("press 1 : Trading \nPress 2 : Forgot pincode ");
        System.out.println("Press 3 : Change mobile number ");
        System.out.println("Press 4 : Print Data sheet \nPress 5 : Exit");
    }

    // function for creating a new Account 
    public static int acCreater(int account_no_counter, Trading t[]) {
        // creating object of Scanner class
        Scanner sc = new Scanner(System.in);

        System.out.print("yes to continue with previous account ");
        System.out.println("or no to create a new account (yes/no) : ");

        if (sc.next().equals("yes") && account_no_counter > 0) {
            System.out.println("Please Enter your Account no.");
            account_no_counter = sc.nextInt();
        } else {
            if (account_no_counter == 0)
                System.out.println("YOU DON'T HAVE ANY ACCOUNT");
            System.out.print("To create a new account, \nPlease Enter your Full Name : ");
            String name = sc.nextLine();
            String name2 = sc.nextLine();
            System.out.print("Please create a PINCODE and REMEMBER it! : ");
            int pincode = sc.nextInt();
            System.out.print("Please Enter your Mobile number : ");
            String mobile_no = sc.next();
            account_no_counter++;
            t[account_no_counter] = new Trading();
            t[account_no_counter].save(name2, mobile_no, pincode);
            System.out.println("Dear, " + name2 + ", YOU ARE SUCCESSFULLY LOGGED IN !");
        }
        return account_no_counter;
    }

    // the main Vyapaar method
    public static void Vyapaar(int a, Trading t1) {

        // creating object of Scanner class
        Scanner sc = new Scanner(System.in);

        // switch on main options
        switch (a) {
            case 1: {
                t1.Print_Tradings();
                int trade_menu = sc.nextInt();

                // switch on trading options
                switch (trade_menu) {

                    // for printing account balance
                    case 1: {
                        System.out.println("Plese enter your pin");
                        if (t1.check_pin(sc.nextInt())) 
                            t1.print_balance();
                        else
                            System.out.println("\n WRONG PIN \n");

                        break;
                    }

                    // for buying stocks 
                    case 2: {
                        System.out.println("Plese enter your pin");
                        if (t1.check_pin(sc.nextInt())) {
                            try {
                                t1.buy();
                            } catch (InsufficientBalanceException e) {
                                System.out.println(e);
                            }
                        } else
                            System.out.println("\n WRONG PIN \n");

                        break;
                    }

                    // for selling stocks
                    case 3: {
                        System.out.println("Plese enter your pin");
                        if (t1.check_pin(sc.nextInt())) 
                            t1.sell();
                        else
                            System.out.println("\n WRONG PIN \n");
                        break;
                    }

                    // for withdrawing money
                    case 4: {
                        System.out.println("Plese enter your pin");
                        if (t1.check_pin(sc.nextInt())) {
                            System.out.println("How much money do you want to withdraw");
                            try {
                                t1.withdraw(sc.nextInt());
                            } catch (InsufficientBalanceException e) {
                                System.out.println(e);
                            }
                        } else
                            System.out.println("\n WRONG PIN \n");
                        break;
                    }

                    // for depositing money
                    case 5: {
                        System.out.println("Plese enter your pin");
                        if (t1.check_pin(sc.nextInt())) {
                            System.out.println("How much money do you want to deposit");
                            try {
                                t1.deposit(sc.nextInt());
                            } catch (DepositLimitException o) {
                                System.out.println(o);
                            }
                        } else
                            System.out.println("\n WRONG PIN \n");      
                        break;
                    }

                    // for switching to mainoptions
                    case 6: {
                        MainOptions();
                        Vyapaar(sc.nextInt(), t1);
                        return;
                    }
                    case 7: return;
                }
                break;
            }

            // case 2 of main options : change pincode
            case 2: {
                System.out.print("Enter your name : ");
                if(sc.nextLine().equals(t1.name)) {
                    System.out.print("Enter your mobile no. : ");
                    if(sc.nextLine().equals(t1.mobile_number)){
                        System.out.print("Enter new pincode :");
                        t1.pincode = sc.nextInt();
                        System.out.println("**Your pincode has successfully changed**");
                    }
                    else {
                        System.out.println("Incurrect mobile number");
                    }
                } else {
                    System.out.println("Wrong name");
                }
                break;
            }

            // for changing mobile number
            case 3: {
                System.out.print("Enter your name : ");
                if(sc.nextLine().equals(t1.name)) {
                    System.out.print("Enter your pincode : ");
                    if(sc.nextInt()==(t1.pincode)){
                        System.out.print("Enter new mobile number :");
                        t1.mobile_number = sc.next();
                        System.out.println("**Your mobile number has successfully changed**");
                    }
                    else {
                        System.out.println("Incurrect pincode number");
                    }
                } else {
                    System.out.println("Wrong name");
                }
                break;
            }

            // for printing data sheet
            case 4: {
                // databook;
                System.out.print("Enter your mobile no. : ");
                if(sc.nextLine().equals(t1.mobile_number)){
                    System.out.println("Enter your pincode : ");
                    if(sc.nextInt()==t1.pincode) {
                        System.out.print("Datasheet is printing :");
                        t1.printDataSheet();
                    }
                    else {
                        System.out.println("Incurrect mobile number");
                    }
                } else {
                    
                    System.out.println("Wrong name");
                }
                break;
            }

            default: {
                return;
            }
        }
        
        // In the last of process
        System.out.println("Do you want to process further, If yes then press 1 otherwise anything");
        if (sc.next().equals("1")) {
            MainOptions();
            Vyapaar(sc.nextInt(), t1);
        }
        
    }

}

// the main trading class
class Trading extends Accounting {

    // creating object of Scanner class
    Scanner sc = new Scanner(System.in);

    private int myStocks;

    public int get_myStock() {
        return myStocks;
    }

    public void print_mystocks() {
        System.out.println(get_myStock());
    }

    public void set_myStocks(int a) {
        myStocks = a;
    }

    public void Print_Tradings() {
        System.out.println(" Press 1: Check balance \n Press 2: Buy \n Press 3: Sell");
        System.out.println(" Press 4: withdraw \n Press 5: Deposit \n Press 6: Main menu \n Press 7: Exit");
    }

    // for buying stocks
    public void buy() throws InsufficientBalanceException {

        int stock = (int) (Math.random() * (9980) + 20);
        System.out.println("The value of one Stock is Rs. " + stock);
        System.out.println("How many stocks do you want to buy");
        int noOfStocks = sc.nextInt();

        if (noOfStocks * stock > this.get_balance()) {
            throw new InsufficientBalanceException("InsufficientBalanceException : insufficient balance");
        } else {
            int kharcha = this.get_balance() - noOfStocks * stock;
            this.set_balance(kharcha);
        }
        System.out.println("[YOU HAVE SUCCESSFULLY BOUGHT " + noOfStocks + " stocks ]");
        int maza = this.get_myStock() + noOfStocks;
        this.set_myStocks(maza);
    }

    // for selling stocks
    public void sell() {

        int stock = (int) (Math.random() * (9980) + 20);
        System.out.println("The value of one Stock is Rs. " + stock);
        System.out.println("You have " + this.get_myStock() + " stocks, How many do you want to sell");
        int noOfStocks = sc.nextInt();
        if (noOfStocks <= this.get_myStock()) {
            int kharcha2 = this.get_balance() + noOfStocks * stock;
            this.set_balance(kharcha2);
            System.out.println("[YOU HAVE SUCCESSFULLY SOLD " + noOfStocks + " STOCKS]");
            int maza2 = this.get_myStock() - noOfStocks;
            this.set_myStocks(maza2);
        } else {
            System.out.println("[YOU DON'T HAVE " + noOfStocks + " STOCKS ]");
        }
    }

    // function for printing datasheet
    public void printDataSheet(){
        System.out.println("\t\tName : "+this.name);
        System.out.println("\t\tAccount number : "+this.account_number);
        System.out.println("\t\tMobile Number : "+this.mobile_number);
        System.out.println("\t\tAccount balance : "+this.get_balance());
        System.out.println("\t\tNo. of Stocks : "+this.get_myStock());

    }

}

// Account class to handle accounting processes
class Accounting  {

    protected String name;
    protected String mobile_number;
    protected int pincode;
    protected int account_number;
    public void save(String name, String mobile_number, int pincode) {
        this.name = name;
        this.mobile_number = mobile_number;
        this.pincode = pincode;
        // this.account_number = account_number;
    }


    protected int acc_balance = 10000;

    public int get_balance() {
        return this.acc_balance;
    }

    public void print_balance() {
        System.out.println("your account balance is " + this.acc_balance);
    }

    public void set_balance(int a) {
        this.acc_balance = a;
    }

    public void deposit(int money) throws DepositLimitException {
        if (money > 10000) {
            throw new DepositLimitException("DepositLimitException : more than 10000 at a time");
        }
        this.acc_balance = acc_balance + money;
    }

    public void withdraw(int money) throws InsufficientBalanceException {
        if (acc_balance < money) {
            throw new InsufficientBalanceException("InsufficientBalanceException : insufficient balance");
        } else {
            this.acc_balance = acc_balance - money;
        }
    }

    public boolean check_pin(int a) {
        if (this.pincode == a) {
            return true;
        } else
            return false;
    }
}

// the exception classes
class InsufficientBalanceException extends Exception {
    InsufficientBalanceException(String message) {
        System.out.println(message);
    }
}

// the exception classes
class DepositLimitException extends Exception {
    DepositLimitException(String message) {
        System.out.println(message);
    }
}


\end{lstlisting}


\newpage
\thispagestyle{others}

\section{Output of program}

\begin{center}
\includegraphics[width = 1.5\textwidth]{one.png}
\end{center}
\begin{center}
\includegraphics[width = 1.5\textwidth]{two.png}
\end{center}
\begin{center}
\includegraphics[width = 1.5\textwidth]{three.png}
\end{center}
\begin{center}
\includegraphics[width = 1.5\textwidth]{four.png}
\end{center}
\begin{center}
\includegraphics[width = 1.5\textwidth]{five.png}
\end{center}
\begin{center}
\includegraphics[width = 1.5\textwidth]{six.png}
\end{center}

\newpage
\thispagestyle{others}
\section{profiling of program}
\begin{center}
\includegraphics[width = 1.0\textwidth]{first.jpeg}
\end{center}
\begin{center}
\includegraphics[width = 1.0\textwidth]{second.jpeg}
\end{center}
\begin{center}
\includegraphics[width = 1.0\textwidth]{third.jpeg}
\end{center}
\begin{center}
\includegraphics[width = 1.0\textwidth]{fourth.jpeg}
\end{center}


\newpage
\thispagestyle{others}
\section{debugging of program}
\begin{center}
\includegraphics[width = 1.0\textwidth]{d_one.png}
\end{center}
\begin{center}
\includegraphics[width = 1.0\textwidth]{d_two.png}
\end{center}
\begin{center}
\includegraphics[width = 1.0\textwidth]{d_three.png}
\end{center}
\begin{center}
\includegraphics[width = 1.0\textwidth]{d_four.png}
\end{center}
\begin{center}
\includegraphics[width = 1.0\textwidth]{d_five.png}
\end{center}
\begin{center}
\includegraphics[width = 1.0\textwidth]{d_six.png}
\end{center}
\begin{center}
\includegraphics[width = 1.0\textwidth]{d_seven.png}
\end{center}
\begin{center}
\includegraphics[width = 1.0\textwidth]{d_eight.png}
\end{center}

\newpage
\thispagestyle{others}
\section{Miscellaneous Information}
\vspace*{10mm}
\paragraph{Starting Date}-\large 18/11/22
\\

\paragraph{Starting Day}-Friday
\\

\paragraph{Ending Date}-21/11/22
\\

\paragraph{Ending Day}-Monday
\\

\paragraph{Total Time required}- 4 days
\\

\paragraph{Total line of code}- 372 lines 
\\

\paragraph{Total number of functions}- 10 + locals
\\
\paragraph{Language Used}- Java
\\
\paragraph{Profiller used}- Jprofiler
\\
\paragraph{Debugger used}- JDB
\\
\paragraph{Project Title}- Trading Account
\\

\end{document}